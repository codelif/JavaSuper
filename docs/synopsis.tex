\documentclass[12pt,a4paper]{article}
\usepackage[a4paper,margin=1in]{geometry}
\usepackage{enumitem} % for compact lists
\begin{document}

\begin{center}
    \Large \textbf{Synopsis of Java Lab Project} \\
    \vspace{0.5cm}
    \large \textbf{E-Commerce Management System} \\
    \vspace{0.5cm}
\end{center}

\noindent
\textbf{Department:} Department of Computer Science \& IT \\
\textbf{Faculty Guide:} 
\begin{itemize}[nosep,left=0pt]
    \item Dr. Kirti Aggarwal
    \item Dr. Neetu Singh
    \item Dr. Tanvi Gautam
\end{itemize}

\noindent
\textbf{Group Members:}
\begin{itemize}[nosep,left=0pt]
    \item Harsh Sharma (2401030232)
    \item Karvy Singh (2401030234)
    \item Rudra Kumar Singh (2401030237)
\end{itemize}

\noindent
\textbf{Batch:} B5

\vspace{0.5cm}

\section*{1. Introduction}
The \textbf{E-Commerce Management System} is a desktop application built using Java Swing. It allows users to browse products, add them to a shopping cart, and place orders through a simple graphical interface. The main aim of this project is to demonstrate how \textbf{Object Oriented Programming (OOP)} concepts can be applied in building a real-world style application. The project provides hands-on experience with encapsulation, inheritance, polymorphism, and abstraction while also strengthening GUI programming skills in Java.

\section*{2. Problem Statement}
Managing a collection of products, maintaining a cart, and handling orders in a structured way can become complicated without a proper system. The problem is to design and implement a simple application that provides the following features:
\begin{itemize}[nosep]
    \item Display a catalog of products with details.
    \item Allow users to add or remove products in a shopping cart.
    \item Enable users to view and confirm their orders.
    \item Provide a basic but user-friendly graphical interface.
\end{itemize}

\section*{3. Proposed Solution / Methodology}
The project will be implemented using \textbf{Java and Swing UI}. The solution will be structured into classes and objects that represent entities such as \texttt{Product}, \texttt{Cart}, \texttt{Order}, and \texttt{User}. The design will follow OOP principles:
\begin{itemize}[nosep]
    \item \textbf{Encapsulation:} Data such as product details and order information will be hidden inside classes with controlled access through getters and setters.
    \item \textbf{Inheritance:} A generic class hierarchy will be created, for example, a base \texttt{User} class and specialized classes such as \texttt{Customer}.
    \item \textbf{Polymorphism:} Functions like displaying product details will be handled differently depending on the type of product or user interface component.
    \item \textbf{Abstraction:} Abstract classes and interfaces will be used for defining generic behaviors like order processing.
\end{itemize}

The GUI will be created using Swing components such as \texttt{JFrame}, \texttt{JPanel}, \texttt{JTable}, and \texttt{JButton}. The interaction between UI and backend logic will demonstrate modular programming practices.

\section*{4. Expected Outcome}
The expected outcome of the project is a functional desktop application where users can:
\begin{itemize}[nosep]
    \item Browse through a list of products.
    \item Add or remove items from their shopping cart.
    \item View and confirm their orders.
\end{itemize}
From an academic perspective, the project will illustrate the use of OOP in designing real-world applications and strengthen our understanding of Java concepts.

\section*{5. Timeline / Work Plan}
\begin{itemize}[nosep]
    \item \textbf{Week 1--2:} Requirement analysis, class design, and UML diagrams.
    \item \textbf{Week 3--4:} Implementation of core classes (\texttt{Product}, \texttt{Cart}, \texttt{Order}).
    \item \textbf{Week 5--6:} Development of GUI using Swing components.
    \item \textbf{Week 7:} Integration of modules and testing.
    \item \textbf{Week 8:} Documentation and final submission.
\end{itemize}

\section*{6. References}
\begin{enumerate}[nosep,left=0pt]
    \item Herbert Schildt, \textit{Java: The Complete Reference}, McGraw Hill.
    \item Oracle Java Documentation: \texttt{https://docs.oracle.com/javase/}
\end{enumerate}

\end{document}
